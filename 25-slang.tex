\begin{frame}{开源社区常见黑话}
	中英文的开源社区都有一些常见的单词缩写或者代称;它们是网络用语或者开源文化的一部分,了解这些说法有助于看懂
	邮件列表、GitHub Issues以及某些群聊中的一些讨论。这里仅举数例,并不具有代表性:
	\vfill
	\begin{description}
		\item[BTW] 即“By The Way”,意为“顺便说一下”;
		\item[RTFM] 即“Read The Fucking Manual”,意为“滚去读文档”;
		\item[OTOH] 即“On The Other Hand”,意为“从另一方面来看”;
		\item[AFAIK] 即“As Far As I Know”,意为“就我所知”;
		\item[GSoC] 即“Google Summer of Code”(Google编程之夏)是由Google公司所主办的年度开源程序设计项目,项目常年涉及各大开源软件项目并向其贡献代码故有一定的知名度;
		\item[xkcd] 指 \href{https://xkcd.com}{xkcd.com} 网站上的连环漫画,其中一部分与开源文化和网络文化有关。
				
	\end{description}
\end{frame}

\begin{frame}{开源社区常见黑话(中文部分)}
	中文社区常见的黑话一般较为随意,这里姑且列出博君一笑:
	\vfill
	\begin{description}
		\item[“底裤”] 有时候指代\texttt{systemd};
		\item[“菊苣”] 类似“大佬”,指代软件核心开发者、主要贡献者或技术大牛;
		\item[“(”] 常见于非正式场合,可理解为缓和语气用的标点符号;
		\item[uuu] 指代Ubuntu,因其名称中有三个字母U;
		\item[滚动/滚挂/回滚] “滚动”指某些滚动Linux发行版日常进行的软件升级操作,“滚挂”指升级系统出错失败,“回滚”即指代英文的“rollback”;
		
	\end{description}
\end{frame}