% 24-slang-debian.tex
\begin{frame}{Debian常用缩写}
	Debian社区语境下有很多不太通用但时常出现的英文缩写,初看可能无法理解其含义,但在社区内工作时需要有一定了解。
	\begin{description}
		\item[DD] 即“Debian Developer”,Debian社区正式成员,其PGP Key登记在案,有内部投票权、被选举权、可以无限制向Debian软件仓库上传已有软件的新版本、有内网机器SSH和LDAP帐号等;
		\item[DM] 即“Debian Maintainer”,其PGP Key登记在案,只能对某些特定的软件无限制地上传新版本(特定软件上传权限由其他Debian Developer授予),除此之外无任何权利;
		\item[Salsa] 即 \texttt{\href{https://salsa.debian.org/}{salsa.debian.org}} 站点,是Debian自建的GitLab CE实例;
		\item[DDPO] 即“Debian Developer's Package Overview”,可以在一个网页上显示某个维护者名下或者某个团队名下所有的软件包列表、健康程度和版本信息。页面位于 \href{https://qa.debian.org/developer.php}{qa.debian.org/developer.php}。\footnote{https://wiki.debian.org/DDPO}

		
	\end{description}
\end{frame}

\begin{frame}{Debian常用缩写(二)}
	\begin{description}
		\item[WNPP] 即“Work-Needing and Prospective Packages”,有一个信息聚合页面来显示无人维护、需要帮助和需要打包的软件包。站点是 \href{https://wnpp.debian.net}{wnpp.debian.net}。\footnote{https://www.debian.org/devel/wnpp/}
		\item[RFA] 即“Request For Adoptation”,通常是某位维护者对某个软件包的维护已力不从心,通过RFA请求其他人接手维护软件包。
		\item[ITP] 即“Intent To Package”,是打算打包某个软件的声明;
		\item[RFS] 即“Request For Sponsorship”,已经完成某个软件的打包需要某位Debian开发者协助审查并上传至官方仓库(Debian Archive,不是Debian所使用的GitLab仓库);
		
	\end{description}
\end{frame}