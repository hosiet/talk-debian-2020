% 40-debian-dev.tex

\begin{frame}{Debian是如何开发的?}
	作为一款完整功能的Linux操作系统,Debian和衍生发行版不同,并没有所谓“上游发行版”。了解Debian的开发工作流有益于了解一个操作系统应当如何独立存在并维持。
	
	\vfill
	
	最最核心的内容无非如下三点:
	
	\begin{itemize}
		\item 以软件包(deb包)为核心:从系统内核到软件库再到各种软件,所有系统文件均由\texttt{.deb}软件包提供;
		\item Debian开发者和软件包维护者(即打包人员)从软件原作者处获取源代码,结合自己编写的打包指令便可使用打包工具链生成新的\texttt{.deb}包;
		\item 打包人员在本地测试通过后将软件源代码和打包指令数字签名,以\textbf{源码包}的形式上传\footnote{请注意,默认情况下是不会上传\texttt{.deb}包的。}至\texttt{Debian FTP Master}\footnote{请将其理解为“文件服务器”,FTP的名字是历史遗留下来的。}服务器,服务器端会自动验证数字签名、权限并为不同硬件架构构建\texttt{.deb}包、将构建成果放入软件源并随镜像网络分发至全球各地。
	\end{itemize}
	
\end{frame}

\begin{frame}{Debian是如何开发的?(二)}{初次引入软件包,以及后续维护}
	初次引入新软件包需要由\texttt{Debian FTP Masters}团队进行审查,确保软件许可证符合要求(是自由的)且软件质量达标;初次审查通过后,后续更新软件包不再需要\texttt{FTP Masters}审查,可以由DD自行上传更新。
	
	\vfill
	
	日常工作:监视上游新版本并进行打包,修复软件包存在的Bug。
	\begin{description}
		\item[监视上游新版本] 只要在打包时正确编写了\texttt{debian/watch}文件,稍后在Debian DDPO和Debian Maintainer Dashboard站点上会自动提示新版本需要打包;
		\item[修复Bug] 软件包列明的维护者会以电子邮件形式收到用户提交的Bug报告,维护者需要相应进行处理。所有Bug报告也可在\url{tracker.debian.org}或\url{bugs.debian.org}网页上查询到。Bug提交者、软件包维护者和所有其他人士均可以回复电子邮件的形式在Bug报告下留言。
	\end{description}
\end{frame}

\begin{frame}{Debian是如何开发的?(三)}{软件包的团队维护}
	某一类软件包可能有相似的特性,它们通常由一个专门存在的团队进行维护。这里稍举几例:
	
	\vfill
	
	\begin{multicols}{2}
		\begin{itemize}
		    \item Debian GCC Maintainers
		    \item Debian Multimedia Maintainers
		    \item Debian X Strike Force
	    	\item Debian Wine Party
		    \item Debian Fonts Task Force
		    \item Debian Python Modules Team
		    \item Debian Rust Maintainers
	    \end{itemize}
    	\columnbreak
    	\begin{itemize}
    		\item Debian Java Maintainers
    		\item Debian Deepin Packaging Team
    		\item Debian Desktop Themes Team
    		\item Debian Chinese Team
    		\item Debian Cloud Team
    		\item Debian Qt and KDE Team
    		\item Debian+Ubuntu MATE Packaging Team
    	\end{itemize}
    \end{multicols}
    

    如需修改团队旗下的软件包,与团队取得联系是很重要的;一般这些团队都有属于自己的邮件列表和GitLab团队,可以通过查看邮件列表存档的活跃人员或者GitLab团队成员名单了解应当与谁联系。
\end{frame}