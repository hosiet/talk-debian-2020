% 70-l10n-i18n.tex

\begin{frame}{Linux桌面本地化的参与入口}
    由于时间有限,这里只给出一些参考链接:
    \begin{description}
    	\item[GNOME] 基于PO文件翻译,使用 \href{https://l10n.gnome.org}{GNOME Damned Lies (l10n.gnome.org)} 自建网站平台,有社区中文团队。
    	\item[KDE] 基于PO文件翻译,中文翻译非正式地托管在了CrowdIn平台上,详情请见 \href{https://community.kde.org/KDE\_Localization/zh-cn}{https://community.kde.org/KDE\_Localization/zh-cn},有社区中文团队。
    	\item[TP] 很多传统命令行工具的翻译托管在TranslationProject(TP)网站上,使用电子邮件提交PO文件翻译(成为提交人员需要申请)。请见 \href{https://translationproject.org/team/zh\_CN.html}{https://translationproject.org/team/zh\_CN.html}。
    	\item[其它] 其它各大软件项目或是直接在Git/SVN仓库中管理翻译文件(PO文件或\texttt{.ts}文件),或是使用如Weblate或CrowdIn等平台托管翻译等等。
    \end{description}

    GNOME/KDE中文社区都有微信群和邮件列表进行组织协调。
\end{frame}