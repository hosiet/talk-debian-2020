%&XeLaTeX
%
% Copyright 2020  Boyuan Yang <byang@debian.org>
% All Rights Reserved.
%
% Build HOWTO
% -----------
% * xelatex
% * biber
% * xelatex
% * xelatex
%
\documentclass[aspectratio=169]{beamer}
\usetheme{Madrid}
\usecolortheme{beaver}
\usepackage{ctex}
\usepackage{booktabs}
\usepackage{ulem}
\usepackage{pgfpages}
\usepackage{hyperref}
\usepackage{svg}
\hypersetup{
    colorlinks=true,
}
\usepackage{graphicx}
\usepackage{tikz}
\usetikzlibrary{arrows, shapes, chains, graphs}
\usepackage[
    %style=authoryear-icomp,
    backend=biber,
    %sortlocale=zh_CN,
    url=true,
    %maxbibnames=2,
]{biblatex}
\usepackage{filecontents}
\usepackage{subfiles}

\addbibresource{debianrefs.bib}
% set global background; must in preamble
\usebackgroundtemplate%
{%
  \tikz \node[opacity=0] {\includegraphics[width=\paperwidth]{openlogo-nd.pdf}};
}%
\setbeamertemplate{bibliography item}[triangle]
\setbeamertemplate{note page}[plain]
%\setbeameroption{show notes on second screen=right}
\setbeameroption{hide notes}
\title[Intro to Debian and FLOSS community]{Debian及自由软件社区简介}
\author[@byang/@hosiet]{杨博远 \\ (Boyuan Yang) \\ \href{mailto:byang@debian.org}{byang@debian.org} }
\institute[Debian]{Debian Project}
\date{\today}
\begin{document}
	
%% Title page
\begin{frame}
\titlepage
\end{frame}

%% Personal introduction
\begin{frame}{个人简介}
	\begin{itemize}
		\item \texttt{Boyuan Yang \href{mailto:byang@debian.org}{<byang@debian.org>}}
		\item Debian开发者(\texttt{“Debian Developer”},2018年至今)
		\item 主要关注方向: \\
		\begin{itemize}
			\item 输入法相关软件(fcitx、ibus)
			\item 软件的国际化和本地化(\texttt{i18n/l10n})
			\item Deepin相关软件
			\item Python软件库与Python所编写软件的打包
			\item 整体质保
		\end{itemize}
	\end{itemize}
    \vfill
\end{frame}


\begin{frame}{目录}
\tableofcontents
\end{frame}


%----------------------
% Presentation contents
%----------------------

\section{Debian简介与社区文化}

\subsection{Debian简介}
\subfile{01-debian-intro}

\subsection{社区文化与特点}
\subfile{02-debian-intro-2}

\subfile{24-slang-debian}

\section{Debian是如何开发的?}

\subfile{40-debian-dev}

\subsection{正确理解Git打包仓库(Salsa GitLab)}

\subfile{47-salsa}

\subsection{有关DD/DM身份}
\subfile{49-be-dd-dm}

\section{与开源社区协作的方法、技巧、注意事项}

\subsection{常用缩写和“黑话”}
\subfile{25-slang}

\subsection{社区协作技巧}
\subfile{30-floss-tricks}

\section{Linux桌面软件本地化情况简介}

\subfile{70-l10n-i18n}

\section*{参考}
\begin{frame}[t]{参考}{项目资源}
\begin{itemize}
\item 官方网站:\url{https://www.debian.org}
\item Debian GitLab (Salsa):\url{https://salsa.debian.org}
\item Debian政策手册:\url{https://www.debian.org/doc/debian-policy/}
\item Debian开发者参考:\url{https://www.debian.org/doc/manuals/developers-reference/}
\item Debian 维基上对下游发行版的指导内容:\url{https://wiki.debian.org/Derivatives/Guidelines}
\item Debian维基上对上游软件作者的指导内容:\url{https://wiki.debian.org/UpstreamGuide}
\end{itemize}
\end{frame}

\iffalse
\begin{frame}{参考}{进一步阅读}
\nocite{*}
\printbibliography
\end{frame}
\fi

\section{Q&A问答}

\begin{frame}{Q\&A}{问答}
\begin{center}
{\Large 问~答~环~节}
\\
任何与Debian相关的问题
\end{center}
\end{frame}

\begin{frame}{版权声明}{Copyright Notice}
本文档以\href{https://creativecommons.org/licenses/by-sa/4.0/legalcode}{知识共享~署名-相同方式共享4.0协议(国际)}发布。
\vspace{1em}

This document is published under \href{https://creativecommons.org/licenses/by-sa/4.0/legalcode}{Creative Commons Attribution-ShareAlike 4.0 International License}.
\end{frame}

%----------------------
% Bibliography
%----------------------
\begin{filecontents*}{debianrefs.bib}
@online{debianwikisou,
  author = {the Debian Project},
  title = {Source Only Upload},
  year = 2016,
  url = {https://wiki.debian.org/SourceOnlyUpload},
  urldate = {2016-11-23}
}
@online{minidebconf20161113,
  author = {Chris Lamb and
            Holger Levsen},
  title = {Reproducible builds status update},
  year = 2016,
  url = {https://people.debian.org/~lamby/2016-11-13-MiniDebConfCambridge/},
  urldate = {2016-11-23},
}
@online{debian-reproduciblebuilds-homepage,
  author = {the Debian Project},
  title = {Overview of various statistics about reproducible builds},
  year = 2016,
  url = {https://tests.reproducible-builds.org/debian/reproducible.html},
  urldate = {2016-11-24},
}
@online{debian-reproduciblebuilds-sourcedateepoch,
  author = {Chris Lamb},
  title = {SOURCE\_DATE\_EPOCH specification},
  year = 2015,
  url = {https://reproducible-builds.org/specs/source-date-epoch/},
}
\end{filecontents*}

%----------------------
% EOF
%----------------------
\end{document}

