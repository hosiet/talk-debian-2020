%&XeLaTeX
%
% Copyright 2020  Boyuan Yang <byang@debian.org>
% All Rights Reserved.
%
% Build HOWTO
% -----------
% * xelatex
% * biber
% * xelatex
% * xelatex
%
\documentclass{beamer}
\usetheme{Madrid}
\usecolortheme{beaver}
\usepackage{ctex}
\usepackage{booktabs}
\usepackage{ulem}
\usepackage{pgfpages}
\usepackage{hyperref}
\usepackage{svg}
\hypersetup{
    colorlinks=true,
}
\usepackage{graphicx}
\usepackage{tikz}
\usetikzlibrary{arrows, shapes, chains, graphs}
\usepackage[
    %style=authoryear-icomp,
    backend=biber,
    %sortlocale=zh_CN,
    url=true,
    %maxbibnames=2,
]{biblatex}
\usepackage{filecontents}
\usepackage{subfiles}

\addbibresource{debianrefs.bib}
% set global background; must in preamble
\usebackgroundtemplate%
{%
  \tikz \node[opacity=0] {\includegraphics[width=\paperwidth]{openlogo-nd.pdf}};
}%
\setbeamertemplate{bibliography item}[triangle]
\setbeamertemplate{note page}[plain]
%\setbeameroption{show notes on second screen=right}
\setbeameroption{hide notes}
\title[Intro to Debian and FLOSS community]{Debian及自由软件社区简介}
\author[@byang/@hosiet]{杨博远 \\ (Boyuan Yang) \\ \href{mailto:byang@debian.org}{byang@debian.org} }
\institute[Debian]{Debian Project}
\date{\today}
\begin{document}
	
%% Title page
\begin{frame}
\titlepage
\end{frame}

%% Personal introduction
\begin{frame}{个人简介}
	\begin{itemize}
		\item \texttt{Boyuan Yang \href{mailto:byang@debian.org}{<byang@debian.org>}}
		\item Debian开发者(\texttt{“Debian Developer”},2018年至今)
		\item 主要关注方向: \\
		\begin{itemize}
			\item 输入法相关软件(fcitx、ibus)
			\item 软件的国际化和本地化(\texttt{i18n/l10n})
			\item Deepin相关软件
			\item Python软件库与Python编写软件的打包
			\item 整体质保
		\end{itemize}
	\end{itemize}
    \vfill
\end{frame}

\subfile{24-slang-debian}
\subfile{25-slang}

\subfile{30-floss-tricks}

\begin{frame}{内容概览}
\tableofcontents
\end{frame}


%----------------------
% Presentation contents
%----------------------

\section{问题提出}
\subsection{存在问题}
\begin{frame}{Debian软件现状}{为了安全,已经做到了什么?}
作为开源软件项目,Debian提供的软件有以下的特性保证安全:
\begin{itemize}
\item 使用GPG签名验证源代码可信与否(需要上游支持);
\item 提供编译选项进行安全增强;
\item 提供经过签名的源码包供用户自行编译使用;
\item 对二进制包所在软件仓库\footnote{默认情况下并没有对单独的二进制.deb包签名,这是一个问题。}签名确保安全性;
\item 部分镜像站点启用HTTPS确保传输安全。
\end{itemize}
\pause
\vskip 1em
存在一个盲点:无法确保从\emph{源代码}到所构建\emph{二进制}软件包的对应关系与安全性。
\end{frame}


\section{工作重点}
\subsection{核心问题}
\subsection{问题分类}
\begin{frame}{可重复构建面临的挑战}{不确定性因素的分类}
要排除不确定性因素,包括以下各个方面:

\vspace{1em}
\begin{enumerate}
\item 时间戳;
\item 时区与地区格式;
\item 不确定的文件排序;
\item 不确定的字典、散列值排序;
\item 用户、用户组、\texttt{umask}、环境变量;
\item 构建路径(build path);\footnote{在Debian 9中容忍同样的构建路径,在sid中则将其纳入考虑范围内。\vspace{1em}}
\item 其它与环境相关的可变情况。
\end{enumerate}
\end{frame}
\subsection{工作思路}
\begin{frame}{Debian中开展可重复构建工作}{工作思路}
\begin{itemize}
\item {在某个Debian环境中连续两次构建一个软件包,改变如下变量:
  \begin{enumerate}
  \item 时间、日期;
  \item 主机名、域名;
  \item 文件系统(disorderfs?);
  \item 时区、地区locale;
  \item 用户ID、用户组ID;
  \item 内核版本、CPU厂商;
  \item \texttt{/bin/sh}默认shell;
  \item 各类其它环境变量,等等。
  \end{enumerate}
  }
\item {比较两者构建结果是否相同:
  \begin{itemize}
    \item 如相同,记录结果;
    \item 如不同,使用特定工具(diffoscope)检查构建结果,寻找不同点并找到原因所在,记录并提交缺陷报告;
    \item 与上游一起工作,提交补丁,等等。
  \end{itemize}
}
\end{itemize}
\end{frame}
\subsection{辅助工具}

\begin{frame}[t]{工作成果展示}{制定标准}
{\Large SOURCE\_DATE\_EPOCH:\xout{\sout{独立自主}}制定的环境变量标准}

\vspace{1em}
\begin{itemize}
\item 纯数字的UNIX时间戳;
\item 如果环境变量中给定了这个变量,则将所有自己程序的时间戳设为该变量对应的时间。
\item 已经被包括\texttt{gcc}在内的各大项目所接受,从根本上解决了时间戳问题。
\end{itemize}
\end{frame}
\begin{frame}[t]{工作成果展示}{与上游的合作}
团队向各种上游发了各式各样的缺陷报告与补丁。略举数例:

\vspace{1em}
\begin{itemize}
\item \texttt{gcc}的\texttt{\_\_DATE\_\_}与\texttt{\_\_TIME\_\_}宏采用\texttt{SOURCE\_DATE\_EPOCH}来防止时间戳问题;
\item \texttt{python} 3.6开始提供有序字典类型;\footnote{\url{https://mail.python.org/pipermail/python-dev/2016-September/146327.html}}
\item \TeX{}Live 2016开始,除了Lua\TeX{}和原始\TeX{}以外均支持\texttt{SOURCE\_DATE\_EPOCH};\footnote{\url{https://www.preining.info/blog/2016/06/tex-live-2016-released/}}
\item \texttt{doxygen}时间戳问题。\footnote{\url{https://bugs.debian.org/792201}}
\end{itemize}
\end{frame}
\subsection{工作进度}

\section{其他项目}
\begin{frame}{其他参与可重复性构建的软件项目}
\begin{columns}
\begin{column}{.5\textwidth}
\begin{itemize}
  \item Arch Linux
  \item Bitcoin
  \item coreboot
  \item Debian
  \item F-Droid
  \item Fedora
  \item FreeBSD
\end{itemize}
\end{column}
\begin{column}{.5\textwidth}
\begin{itemize}

  \item Guix
  \item LEDE
  \item NetBSD
  \item OpenWrt
  \item openSUSE
  \item Tails
  \item Tor Browser
\end{itemize}
\end{column}
\end{columns}

\vfill
\begin{center}
……还有更多!
\end{center}
\end{frame}
\section{未来展望}
\begin{frame}{未来展望}{项目带来的额外优势}
使用持续集成方式探测检查软件包的可重复构建性,还可能带来以下额外好处:
\begin{enumerate}
\item 将\texttt{buildd}的一部分构建任务交给持续集成系统,减轻负担;
\item 承担一部分质保任务,探测软件包regression;
\item 检查出软件在边界条件上可能出现的缺陷;
\item 避免不确定性因素对软件质量的影响(互联网质量、时间戳等)。
\end{enumerate}
\vfill
\end{frame}
\section*{参考}
\begin{frame}[t]{参考}{项目资源}
\begin{itemize}
\item 官方网站:\url{https://reproducible-builds.org}
\item Debian测试情况:\url{https://reproducible.debian.net}
\item Debian Wiki主页:\url{https://wiki.debian.org/ReproducibleBuilds}
\item 项目每周报告:\url{https://reproducible.alioth.debian.org/blog/}
\item 项目历史记录:\url{https://wiki.debian.org/ReproducibleBuilds/History}
\end{itemize}
\end{frame}
\begin{frame}{参考}{进一步阅读}
\nocite{*}
\printbibliography
\end{frame}

\begin{frame}{Q\&A}{问答}
\begin{center}
{\Large 问~答~环~节}
\end{center}
\end{frame}

\begin{frame}{版权声明}{Copyright Notice}
本文档以\href{https://creativecommons.org/licenses/by-sa/4.0/legalcode}{知识共享~署名-相同方式共享4.0协议(国际)}发布。
\vspace{1em}

This document is published under \href{https://creativecommons.org/licenses/by-sa/4.0/legalcode}{Creative Commons Attribution-ShareAlike 4.0 International License}.
\end{frame}

%----------------------
% Bibliography
%----------------------
\begin{filecontents*}{debianrefs.bib}
@online{debianwikisou,
  author = {the Debian Project},
  title = {Source Only Upload},
  year = 2016,
  url = {https://wiki.debian.org/SourceOnlyUpload},
  urldate = {2016-11-23}
}
@online{minidebconf20161113,
  author = {Chris Lamb and
            Holger Levsen},
  title = {Reproducible builds status update},
  year = 2016,
  url = {https://people.debian.org/~lamby/2016-11-13-MiniDebConfCambridge/},
  urldate = {2016-11-23},
}
@online{debian-reproduciblebuilds-homepage,
  author = {the Debian Project},
  title = {Overview of various statistics about reproducible builds},
  year = 2016,
  url = {https://tests.reproducible-builds.org/debian/reproducible.html},
  urldate = {2016-11-24},
}
@online{debian-reproduciblebuilds-sourcedateepoch,
  author = {Chris Lamb},
  title = {SOURCE\_DATE\_EPOCH specification},
  year = 2015,
  url = {https://reproducible-builds.org/specs/source-date-epoch/},
}
\end{filecontents*}

%----------------------
% EOF
%----------------------
\end{document}

